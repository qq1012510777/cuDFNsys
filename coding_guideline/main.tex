

\documentclass{article}  
\bibliographystyle{unsrt}

\usepackage{bm}
\usepackage{graphicx}
\usepackage{subfigure}
\usepackage{amsmath}
\usepackage{esint}
\usepackage{soul}
\usepackage{listings}
\usepackage{xcolor}


\lstset{language=C++}%

\lstset{breaklines}%

\lstset{extendedchars=false}%
\lstset{                        %Settings for listings package.
	language=[ANSI]{C++},
	%backgroundcolor=\color{lightgray},
	basicstyle=\footnotesize,
	breakatwhitespace=false,
	breaklines=true,
	captionpos=b,
	commentstyle=\color{olive},
	directivestyle=\color{blue},
	extendedchars=false,
	% frame=single,%shadowbox
	framerule=0pt,
	keywordstyle=\color{blue}\bfseries,
	morekeywords={*,define,*,include...},
	numbersep=5pt,
	rulesepcolor=\color{red!20!green!20!blue!20},
	showspaces=false,
	showstringspaces=false,
	showtabs=false,
	stepnumber=2,
	stringstyle=\color{purple},
	tabsize=4,
	title=\lstname
}


\title{C++ coding guideline for $cuDFNsys$}
\author         {Tingchang YIN\\Zhejiang University\\Westlake University}

%\date{\today}  
   

\begin{document}
\maketitle
\newpage
\tableofcontents
\newpage
\section{File structures}
Let us denote the first level file by *, second level by + and third level by \^.
Hierarchy of $cuMechcsysDFN$ should be as follows 

\begin{lstlisting}
	* include
		+ HeaderFile1
			^ Foo1.cuh
		+ HeaderFile2
			^ Foo2.cuh
	* src
		+ srcFile1
			^ Foo1.cu
		+ srcFile2
			^ Foo2.cu
	* sandbox1
		+ src
			^ main.cu
		+ bin
			^ main
		+ build
			^ makefile
		+ CMakeLists.cmake
		+ CompileCode.sh
	* sandbox2
		+ src
			^ main.cu
		+ bin
			^ main
		+ build
			^ makefile
		+ CMakeLists.cmake
		+ CompileCode.sh
\end{lstlisting}

\section{Namespace}
Namespace is $cuDFNsys$.

\section{Separating source members}
Definitions and implementations of classes/structs/functions should be contained in separated files (i.e. header and source files). All words of the names of files should begin with a \textbf{“capital”} letter.For example, 
\begin{itemize}
	\item <FractureRadius.cuh>
	\item <FractureRadius.cu>
\end{itemize}
\begin{lstlisting}
	//<FractureRadius.cuh>
	#pragma once
	#include <iostream>
	using namespace std;
	class FractureRadius
	{
		public:
			float Radius;
		public:
			FractureRadius(const uint i);	
			float SecondMoments();
	};
\end{lstlisting}
\begin{lstlisting}
	//<FractureRadius.cu>
	#include "FractureRadius.cuh"
	FractureRadius::FractureRadius(const uint i)
	{
		this->Radius = i * 10;
	};
	
	float FractureRadius::SecondMoments()
	{
		return pow(this->Radius, 2);
	};
\end{lstlisting}

\section{Standard comment headers}
In header files, we should include the following information
\begin{lstlisting}
	///////////////////////////////////////////////////////////////////
	// NAME:	          FooBar.h
	//
	// PURPOSE:	          Definition of FooBar class. This class is
	//	                  responsible for doing FooBar stuff.
	//
	// FUNCTIONS/OBJECTS: CFooBar
	//
	// AUTHOR:	          James
	///////////////////////////////////////////////////////////////////	
\end{lstlisting}

\section{Class/struct/function naming}
All words of a class/struct begin with a \textbf{capital} letter. For example:
\begin{lstlisting}
	class Animal;
	struct DriverJames;
\end{lstlisting}

The naming of a function (including a member function of a class) should be based on the \textbf{purpose} or the \textbf{method}. Again, all words begin with a \textbf{capital} letter. For example,
\begin{lstlisting}
	float GetFractureTag(const uint i); 
	// based on the purpose
	void ConjugateGradient(); 
	// based on the method
\end{lstlisting}

\section{Class member organization}
For example:
\begin{lstlisting}
	class Foo : public FooBase
	{
		// methods
		//-------------------------------
		public :
		protected :
		private :
		// attributes
		//-------------------------------
		public :
		protected :
		private :
	};
\end{lstlisting}

\section{Data variable naming}
All words begin with a \textbf{capital} letter. For example,
\begin{lstlisting}
	class Foo
	{
	public:
		int             FooTag;
		string          FooName;
		bool            FooState;
	};
\end{lstlisting}
When calling a member data variable in a class member function, "this->" should be always used. For example
\begin{lstlisting}
	this->Var1 = 1;
\end{lstlisting}

For naming global variables related to tolerance, the format is
\begin{lstlisting}
#define _TOL_IdentifyEleNO 1e-7
__device__ const float _TOL_DetectIntersection = 1e-7;
\end{lstlisting}

\section{Member descriptive comments}
All member data and functions should be prefaced with single line comments describing the purpose of the member. For example:
\begin{lstlisting}
	class Foo
	{
	public:
		// the tag of the Foo
		uint Tag;
	
	public:
		// delete the tag of the foo
		void DeleteTag();
	};
\end{lstlisting}

\section{Descriptive comment of a function}
In source (.cpp) files, a function should be supplemented by detailed descriptive comments. For example:
\begin{lstlisting}
	// ====================================================
	// NAME	       : GetName
	//
	// DESCRIPTION : Accessor for the name of this object.
	// AUTHOR	   : DEV1
	// DATE	       : 10/10/99
	// ====================================================
	void Foo::GetName( CString& strName )
	{
		strName = this->Name;
	} // GetName	
	
\end{lstlisting}
	Also, all functions should include a trailing comment just after the closing brace of the function scope that identifies the function name.

\section{About inline functions}
	Inline function may increase efficiency if it is \textbf{small}.
	
	Remember, inlining is only a request to the compiler, not a command. Compiler can ignore the request for inlining. Compiler may not perform inlining in such circumstances like:
	\begin{itemize}
	\item If a function contains a loop. (for, while, do-while)
	\item If a function contains static variables.
	\item If a function is recursive.
	\item If a function return type is other than void, and the return statement doesn’t exist in function body.
	\item If a function contains switch or goto statement.	
	\end{itemize}

\end{document}